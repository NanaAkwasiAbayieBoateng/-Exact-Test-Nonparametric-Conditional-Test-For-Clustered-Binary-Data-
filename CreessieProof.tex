\documentclass{article}\usepackage[]{graphicx}\usepackage[]{color}
%% maxwidth is the original width if it is less than linewidth
%% otherwise use linewidth (to make sure the graphics do not exceed the margin)
\makeatletter
\def\maxwidth{ %
  \ifdim\Gin@nat@width>\linewidth
    \linewidth
  \else
    \Gin@nat@width
  \fi
}
\makeatother

\definecolor{fgcolor}{rgb}{0.345, 0.345, 0.345}
\newcommand{\hlnum}[1]{\textcolor[rgb]{0.686,0.059,0.569}{#1}}%
\newcommand{\hlstr}[1]{\textcolor[rgb]{0.192,0.494,0.8}{#1}}%
\newcommand{\hlcom}[1]{\textcolor[rgb]{0.678,0.584,0.686}{\textit{#1}}}%
\newcommand{\hlopt}[1]{\textcolor[rgb]{0,0,0}{#1}}%
\newcommand{\hlstd}[1]{\textcolor[rgb]{0.345,0.345,0.345}{#1}}%
\newcommand{\hlkwa}[1]{\textcolor[rgb]{0.161,0.373,0.58}{\textbf{#1}}}%
\newcommand{\hlkwb}[1]{\textcolor[rgb]{0.69,0.353,0.396}{#1}}%
\newcommand{\hlkwc}[1]{\textcolor[rgb]{0.333,0.667,0.333}{#1}}%
\newcommand{\hlkwd}[1]{\textcolor[rgb]{0.737,0.353,0.396}{\textbf{#1}}}%

\usepackage{framed}
\makeatletter
\newenvironment{kframe}{%
 \def\at@end@of@kframe{}%
 \ifinner\ifhmode%
  \def\at@end@of@kframe{\end{minipage}}%
  \begin{minipage}{\columnwidth}%
 \fi\fi%
 \def\FrameCommand##1{\hskip\@totalleftmargin \hskip-\fboxsep
 \colorbox{shadecolor}{##1}\hskip-\fboxsep
     % There is no \\@totalrightmargin, so:
     \hskip-\linewidth \hskip-\@totalleftmargin \hskip\columnwidth}%
 \MakeFramed {\advance\hsize-\width
   \@totalleftmargin\z@ \linewidth\hsize
   \@setminipage}}%
 {\par\unskip\endMakeFramed%
 \at@end@of@kframe}
\makeatother

\definecolor{shadecolor}{rgb}{.97, .97, .97}
\definecolor{messagecolor}{rgb}{0, 0, 0}
\definecolor{warningcolor}{rgb}{1, 0, 1}
\definecolor{errorcolor}{rgb}{1, 0, 0}
\newenvironment{knitrout}{}{} % an empty environment to be redefined in TeX

\usepackage{alltt}
\usepackage{amsmath}
\usepackage{amssymb}
\usepackage{amsfonts}
\usepackage{longtable}
\usepackage{alltt}
\usepackage{setspace,relsize} % for latex(describe()) (Hmisc package) and latex.table
\usepackage{moreverb}
\usepackage[pdftex]{lscape}
\usepackage{lipsum}
\usepackage{tabularx}
\usepackage{tabularx,ragged2e,booktabs,caption}%table package
\newcolumntype{C}[1]{>{\Centering}m{#1}}
\renewcommand\tabularxcolumn[1]{C{#1}}
\usepackage{graphicx}

\usepackage{wrapfig}
\usepackage[utf8]{inputenc}
\usepackage[T1]{fontenc}
\usepackage{lmodern}
\usepackage{listings}
\usepackage{fancyhdr}
\usepackage{fullpage}

\definecolor{MyDarkGreen}{rgb}{0.0,0.4,0.0}


%===========================================================================================
% Numbering Equations
%\numberwithin{equation}{section} %sets equation numbers <chapter>.<section>.<index>
\numberwithin{equation}{subsection} %sets equation numbers <chapter>.<section>.<subsection>.<index>
%\numberwithin{equation}{subsubsection} %sets equation numbers <chapter>.<section>.<subsection>.<subsubsection>.<index>



 %===================================================================================================================

%===================================================================================================================
% Basic Commands for Theorem, Lemma,Proposition,etc.....

\renewcommand{\baselinestretch}{2}
\renewcommand{\arraystretch}{.5}
\newcommand{\qed}{\hfill$\Box$}
\newtheorem{fact}{Theorem}[section]
\newtheorem{claim}{Claim}
\newtheorem{theorem}[fact]{Theorem}
\newtheorem{word}[fact]{Definition}
\newtheorem{prop}[fact]{Proposition}
\newtheorem{ob}[fact]{Observation}
\newtheorem{Corollary}[fact]{Corollary}
\newtheorem{corollary}[fact]{Corollary}
\newtheorem{lemma}[fact]{Lemma}
\newtheorem{Guess}[fact]{Conjecture}
\newtheorem{conj}[fact]{Conjecture}
\def\theotheorem{A\arabic{theorem}}
\newtheorem{mydef}{Definition}
\newtheorem{thm}{Theorem}
\newtheorem{lem}{Lemma}[thm]
\newenvironment{proof}[1][Proof]{\begin{trivlist}
\item[\hskip \labelsep {\bfseries #1}]}{\end{trivlist}}
\newenvironment{definition}[1][Definition]{\begin{trivlist}
\item[\hskip \labelsep {\bfseries #1}]}{\end{trivlist}}
\newenvironment{example}[1][Example]{\begin{trivlist}
\item[\hskip \labelsep {\bfseries #1}]}{\end{trivlist}}
\newenvironment{remark}[1][Remark]{\begin{trivlist}
\item[\hskip \labelsep {\bfseries #1}]}{\end{trivlist}}



%================================================================================
%%%%%%%%%%%commands for problem
%================================================================================
\makeatletter
\newenvironment{problem}{\@startsection
       {section}
       {1}
       {-.2em}
       {-3.5ex plus -1ex minus -.2ex}
       {2.3ex plus .2ex}
       {\pagebreak[3]%forces pagebreak when space is small; use \eject for better results
       \large\bf\noindent{Problem }
       }
       }
       {%\vspace{1ex}\begin{center} \rule{0.3\linewidth}{.3pt}\end{center}}
       \begin{center}\large\bf \ldots\ldots\ldots\end{center}}
\makeatother








\setlength{\oddsidemargin}{0.truein}
\setlength{\evensidemargin}{0.truein}
\setlength{\textwidth}{6.5truein} \setlength{\topmargin}{-.55truein}
\setlength{\textheight}{9.0truein}

%%%%%%%%%%%%%%%%%%%bibliography%%%%%%%%%%%%%%%%%%%%%%%%%%%
\def\thebiblio#1{\list
{[\arabic{enumi}]}{\settowidth\labelwidth{[#1]}\leftmargin\labelwidth
\advance\leftmargin\labelsep
\usecounter{enumi}}
\def\newblock{\hskip .11em plus .33em minus .07em}
\sloppy\clubpenalty4000\widowpenalty4000
\sfcode`\.=1000\relax}
\let\endthebiblio=\endlist
\def\eqalign#1{\null\,\vcenter{\openup\jot\ialign
{\strut\hfil$\displaystyle{##}$&$\displaystyle{{}##}$\hfil\crcr#1\crcr}}\,}
\def\refhg{\hangindent=20pt\hangafter=1}
\def\refmark{\par\vskip 2mm\noindent\refhg}
\def\bLambda{\mbox{\boldmath $\Lambda$}}
\def\bDelta{\mbox{\boldmath $\Delta$}}
\def\btheta{\mbox{\boldmath $\theta$}}
\def\bSigma{\mbox{\boldmath $\Sigma$}}
\def\bbeta{\mbox{\boldmath $\beta$}}
\def\bphi{\mbox{\boldmath $\phi$}}
\def\balpha{\mbox{\boldmath $\alpha$}}
\def\bJ{{\bf J}}
\def\bgamma{\mbox{\boldmath $\gamma$}}
\def\bmu{\mbox{\boldmath $\mu$}}
\def\bY{{\bf Y}}
\def\bU{{\bf U}}
\def\bx{{\bf x}}
\def\bX{{\bf X}}
\def\bW{{\bf W}}
\def\bZ{{\bf Z}}
\def\bz{{\bf z}}
\def\logit{\rm logit}
\def\bgn{\begin{eqnarray*}}
\def\edn{\end{eqnarray*}}
\def\bg{\begin{eqnarray}}



%===========================================================================
% commands for displaying codes in appendix
%============================================================================
\usepackage{listings}
%===========================================================================
% commands for displaying codes in appendix
%============================================================================
\lstloadlanguages{r}%
\lstset{language=r }

% Includes an r script.
% The first parameter is the label, which also is the name of the script
%   without the .m.
% The second parameter is the optional caption.
\newcommand{\matlabscript}[2]
  {\begin{itemize}\item[]\lstinputlisting[caption=#2,label=#1]{#1.r}\end{itemize}}
\IfFileExists{upquote.sty}{\usepackage{upquote}}{}
\begin{document}
\begin{center}
{\Large {\bf Nana Boateng}} \\
\vspace{5mm}
{\Large Department of Mathematics} \\
\vspace{5mm}
%{\Large } \\
\vspace{5mm}
\end{center}
\textbf{Power Divergence Statistics}

Let the random variables $\textbf{Y}=(Y_{1},Y_{2},\cdots,Y_{k})$ form a multinomial distribution with parameters      $\Pi=(\pi_{1},\cdots,\pi_{k})$\\
\begin{equation*}
P_{r}(Y=y)=\frac{n!}{\pi_{1}!,\cdots,\pi_{k}}\pi_{1}^{y_{1}}\pi_{2}^{y_{}}\cdots \pi_{k}^{y_{k}}
\end{equation*}

To test the null hypothesis $H_{0}$ :  $ \Pi=\Pi_{0}$ \hspace{5mm}where\hspace{2mm} $\Pi_{0}=(\pi_{01},\pi_{02},\cdots,\pi_{0k})$\\
Cressie and Read \cite{cressie1984} suggest the power divegence theorem which simplifies to commonly used chi-squared test statistic for testing multinomial distributions  for some $\lambda \in \mathbb{R}$




\begin{equation*}
2nI^{\lambda}\left(\frac{x}{n}:\pi_{0} \right)=\frac{2}{\lambda(\lambda+1)}\sum\limits_{i=1}^{K}y_{ij}\left[\left(\frac{y_{ij}}{E_{ij}}\right)^{\lambda}-1\right]\hspace{3mm}\lambda \in \mathbb{R}
\end{equation*}

Where $\lambda=1,0,-2,-1,-0.5$ and $\frac{2}{3}$ corresponds to the Pearson-Chi-squared statistic, log likelihood ratio statistic,Neyman statistics,Kullback and Leibler,Freeman-Tukey and Cresie -Read respectively.\\



\textbf{The Power Divergence  $2nI^{\lambda}\left(\frac{x}{n}:\Pi_{0} \right) \stackrel{\text{asyp}}{\sim} X^{2}_{k-1}$ under $H_{0}$}
\begin{proof}
For $\lambda \neq 0,-1$\\

\begin{equation*}
2nI^{\lambda}\left(\frac{x}{n}:\Pi_{0} \right)=\frac{2}{\lambda(\lambda+1)}\sum\limits_{i=1}^{K}y_{ij}\left[\left(\frac{y_{ij}}{E_{ij}}\right)^{\lambda}-1\right]\hspace{3mm}\lambda \in \mathbb{R}
\end{equation*}






\begin{equation*}
=\frac{2}{\lambda(\lambda+1)}\sum\limits_{i=1}^{K}n\pi_{0i}\left[\left(\frac{y_{ij}}{n\pi_{0i}}\right)^{\lambda+1}-1\right]
\end{equation*}

\begin{equation*}
=\frac{2n}{\lambda(\lambda+1)}\sum\limits_{i=1}^{K}\pi_{0i}\left[\left(\frac{n\pi_{0i}+y_{ij}-n\pi_{0i}}{n\pi_{0i}}\right)^{\lambda+1}-1\right]
\end{equation*}

\begin{equation*}
=\frac{2n}{\lambda(\lambda+1)}\sum\limits_{i=1}^{K}\pi_{0i}\left[\left(1+\frac{y_{ij}-n\pi_{0i}}{n\pi_{0i}}\right)^{\lambda+1}-1\right]
\end{equation*}





%\begin{equation*}
%\longrightarrow \frac{2}{\lambda(\lambda+1)}\sum\limits_{i=1}^{K}n\pi_{0i}\left[\left(\frac{y_{ij}}{n\pi_{0i}}\right)^{\lambda+1}  -1\right]
%\end{equation*}

%\begin{equation*}
%\longrightarrow \frac{2}{\lambda(\lambda+1)}\sum\limits_{i=1}^{K}y_{ij}\left[\left(\frac{y_{ij}}{n\pi_{0i}}\right)^{\lambda}  -1\right]
%\end{equation*}


%\begin{equation*}
%\longrightarrow \frac{2}{\lambda(\lambda+1)}\sum\limits_{i=1}^{K}y_{ij}\left[\frac{y_{ij}^{\lambda}}{(n\pi_{0i})^{\lambda}}  -1\right]
%\end{equation*}

\begin{equation*}
\longrightarrow \frac{2n}{\lambda(\lambda+1)}\sum\limits_{i=1}^{K}\pi_{0i}\left[(1+v_{i})^{\lambda+1}  -1\right]  \hspace{3mm}\text{where}\hspace{3mm} v_{i}= \frac{y_{ij}-n\pi_{0i}}{n\pi_{0i}}
\end{equation*}

Expandig by Taylor series.

\begin{equation*}
\longrightarrow \frac{2n}{\lambda(\lambda+1)}\sum\limits_{i=1}^{K}\pi_{0i}\left[(\lambda+1)v_{i}+\frac{\lambda(\lambda+1)}{2}v_{i}^{2}+O_{p}(v_{i}^{3})\right]  
\end{equation*}

\begin{equation*}
\longrightarrow 2n\sum\limits_{i=1}^{K}\left[\frac{\pi_{0i}v_{i}}{\lambda}+\frac{\pi_{0i}v_{i}^{2}}{2}+O_{p}(v_{i}^{3})\right]  
\end{equation*}

\begin{equation*}
\longrightarrow 2n\left[\sum\limits_{i=1}^{K}\frac{\pi_{0i}v_{i}^{2}}{2}+O_{p}(\frac{1}{n})\right]  
\end{equation*}


\begin{equation*}
2nI^{\lambda}\left(\frac{y}{n}:\Pi_{0} \right)=2nI^{1}\left(\frac{y}{n}:\Pi_{0} \right) +O_{p}(1),\lambda \in \mathbb{R}
\end{equation*}

Thus the power divergence has the same asymptotic distribution as the Pearson-Chi squared which is known to be $X^{2}_{k-1}$ under $H_{0}$

\end{proof}



\cleardoublepage

%=================================================BIBLIOGRAPHY==============================================================
\newpage
\begin{center}
{\bf REFERENCES}
\end{center}
\addcontentsline{toc}{section}{\rm REFERENCES \dotfill}
\begin{thebiblio}{99}
%---------------------------------------------------------------------------------------------------------------------------

\bibitem{cressie1984}
Cressie Noel,Read R.C Timothy,\emph{Multinomial Goodness of Fit},Journal of the Royal Statistical Society. Series B (Methodological), Vol. 46, No. 3(1984), pp. 440-464



\end{thebiblio}

\end{document}




